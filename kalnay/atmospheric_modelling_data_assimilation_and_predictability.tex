\chapter{Atmospheric Modelling, Data Assimilation and Predictability}
\label{ch:kalnay}

The following constitutes the notes that I have made whilst going through Atmospheric Modeling, Data Assimilation and Predictability\citep{kalnay_atmospheric_2003}.

\section{Historical overview of numerical weather prediction}
\label{sec:kalnay:historical_overview}

\subsection{Introduction}
\label{subs:historical_overview:introduction}

The improvement in skill of numerical weather prediction over the last 40 years is due to the following factors:
\begin{itemize}
\item The increased power of supercomputers - this allows for finer numerical resolution, and means that fewer approximations are required in the operational atmospheric models;
\item The improved representation of small-scale physical processes - these include clouds, precipitation, turbulent transfer processes;
\item The use of more accurate methods of data assimilation - these lead to improved initial conditions for the models;
\item The increased availability of data - for example, satellite and aircraft data over the oceans and the Southern Hemisphere. 
\end{itemize}

\subsection{Early developments}
\label{subs:historical_overview:early_developments}

\subsection{Primitive equations, global and regional models, and nonhydrostatic models}
\label{subs:historical_overview:primitive_equations}

Whilst quasi-geostrophic equations may be useful for understanding the large-scale extratropical dynamics of the atmosphere, they are not accurate enough to allow for continued progress in Numerical Weather Prediction (NWP).
They are therefore replaced by the primitive equations.
The primitive equations are conservation laws applied to individual parcels of air:
\begin{itemize}
\item Conservation of three-dimensional momentum - the equations of motion;
\item Conservation of energy - the first law of thermodynamics;
\item Conservation of dry air mass - the continuity equation;
\item Conservation of moisture in all of its phases;
\item Ideal gas law.
\end{itemize}
Unlike the quasi-geostrophic equation previously used, these equations capture fast gravity and sound waves; however this requires that finer time discretisation is used (or some alternative method).

Continuous equations of motion are solved by discretisation in space and time.
The accuracy of a model is strongly influenced by the spatial resolution: typically, the higher the resolution, the more accurate the model.
This increase in resolution incurs greater computational cost though.
Whilst modern numerical methods have attempted to alleviate this cost (using semi-implicit and semi-Lagrangian time schemes), the continued increase in spatial resolution is effectively an endless ``race to the bottom'', with constant demand for higher resolution to improve NWP forecasts.

\subsection{Data assimilation determination of the initial conditions for the computer forecasts}
\label{subs:historical_overview:assimilation}

\begin{itemize}
    \item NWP is an intial-value problem --- given an estimated starting state, the model forecasts the system's evolution --- as therefore, the determination of the initial conditions is very imporant.
    \item This hass been achieved through:
    \begin{itemize}
        \item successive correction methods (SCM),
        \item optimal interpolation (OI),
        \item variational methods in three and four dimensions (3D-Var, 4D-Var),
        \item Kalman filtering (KF).
    \end{itemize}
    \item Additional information is required to prepare the intial conditions for forecasts, which we call `background', `first guess' or `prior information'.
    \item Terminology for this section:
    \begin{itemize}
        \item $x^b$: a three-dimensional array, `background field',
        \item $y^o$: `observed variables',
        \item $H$: `observation operator' --- performs the necessary interpolation and transformation from model variables to observation space,
        \item $H \left( x^b \right)$: `first guess observations',
        \item $y^o - H \left( x^b \right)$: `observation increments' or `innovations' --- the difference between the observations and the model first guess,
        \item $W$: weights,
        \item $x^a$: `the analysis' --- adding the innovations to the model forecast with weights $W$ based on the estimated statistacl error covariances of forecast and observations:
        \begin{equation}
            \mathbf{x^a} = \mathbf{x^b} + \mathbf{W} 
            \left[ \mathbf{y^o} - \mathbf{H} \left( \mathbf{x^b} \right) \right]
        \end{equation}
        This acts as the basis for various analysis schemes
        \item SCM: weights are empirically determined, as a function of distance between observation and grid point.
        \item OI: matrix of weights is determined from the minimisation of analysis errors at each grid point.
        \item 3D-Var: Defining cost function proportional to the square of the distance between the analysis and the background and observations, and aim to minimise it.
        \item Differences between OI and 3D-Var: in OI weights W are obtained using simplifications; in 3D-Var minimisation is performed directly.
        \item 4D-Var: lookat minimisation problem from 3D-Var but also considering distance to observations over time window.
    \end{itemize}
\end{itemize}

\subsection{Weather predictability, ensemble forecasting, and seasonal to interannual prediction}
\label{subs:historical_overview:predictability}

\begin{itemize}
    \item Given that the skill of forecasting decreases with time, it is suggested that stochastic forecasting should be favoured over deterministic forecasting, providing an estimate of the prediction.
    \item This is achieved through ensemble forecasting, where multiple models are run forward to forecast, with small perturbations being introduced.
    \item The result is an ensemble average forecast that is typically more accurate than individual forecasts beyond the first few days; furthermore, it provides forecasters with a measure of reliability.
    \item This system can also be used to develop adaptive and targeted observation networks.
\end{itemize}

% \section{The continuous equations}
% \label{sec:kalnay:continuous_equations}

% \section{Numerical discretization of the equations of motion}
% \label{sec:kalnay:numerical_discretization}

% \section{Introduction to the parameterization of subgrid-scale physical processes}
% \label{sec:kalnay:parameterization}

\section{Data assimilation}
\label{sec:kalnay:data_assimilation}

\subsection{Introduction}
\label{sub:data_assimilation:intro}

\subsection{Empirical analysis schemes}
\label{sub:data_assimilation:analysis}

\subsection{Introduction to least squares methods}
\label{sub:data_assimilation:squares}

\subsection{Multivariate statistical data asssimilation methods}
\label{sub:data_assimilation:multivar}

\subsection{3D-Var, the physical space analysis scheme (PSAS), and their relation to OI}
\label{sub:data_assimilation:3d-var}

\subsection{Advanced data assimilation methods with evolving forecast error covariance}
\label{sub:data_assimilation:advanced}

\subsection{Dynamical and physical balance in the initial conditions}
\label{sub:data_assimilation:balance}

\subsection{Quality control of observations}
\label{sub:data_assimilation:quality}


% \section{Atmospheric predictability and ensemble forecasting}
% \label{sec:kalnay:predictability}

