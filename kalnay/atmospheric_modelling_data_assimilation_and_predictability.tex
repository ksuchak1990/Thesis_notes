\chapter{Atmospheric Modelling, Data Assimilation and Predictability}
\label{ch:kalnay}

The following constitutes the notes that I have made whilst going through Atmospheric Modeling, Data Assimilation and Predictability\citep{kalnay_atmospheric_2003}.

\section{Historical overview of numerical weather prediction}
\label{sec:kalnay:historical_overview}

\subsection{Introduction}
\label{subs:historical_overview:introduction}

The improvement in skill of numerical weather prediction over the last 40 years is due to the following factors:
\begin{itemize}
\item The increased power of supercomputers - this allows for finer numerical resolution, and means that fewer approximations are required in the operational atmospheric models;
\item The improved representation of small-scale physical processes - these include clouds, precipitation, turbulent transfer processes;
\item The use of more accurate methods of data assimilation - these lead to improved initial conditions for the models;
\item The increased availability of data - for example, satellite and aircraft data over the oceans and the Southern Hemisphere. 
\end{itemize}

\subsection{Early developments}
\label{subs:historical_overview:early_developments}

\subsection{Primitive equations, global and regional models, and nonhydrostatic models}
\label{subs:historical_overview:primitive_equations}

Whilst quasi-geostrophic equations may be useful for understanding the large-scale extratropical dynamics of the atmosphere, they are not accurate enough to allow for continued progress in Numerical Weather Prediction (NWP).
They are therefore replaced by the primitive equations.
The primitive equations are conservation laws applied to individual parcels of air:
\begin{itemize}
\item Conservation of three-dimensional momentum - the equations of motion;
\item Conservation of energy - the first law of thermodynamics;
\item Conservation of dry air mass - the continuity equation;
\item Conservation of moisture in all of its phases;
\item Ideal gas law.
\end{itemize}
Unlike the quasi-geostrophic equation previously used, these equations capture fast gravity and sound waves; however this requires that finer time discretisation is used (or some alternative method).

Continuous equations of motion are solved by discretisation in space and time.
The accuracy of a model is strongly influenced by the spatial resolution: typically, the higher the resolution, the more accurate the model.
This increase in resolution incurs greater computational cost though.
Whilst modern numerical methods have attempted to alleviate this cost (using semi-implicit and semi-Lagrangian time schemes), the continued increase in spatial resolution is effectively an endless ``race to the bottom'', with constant demand for higher resolution to improve NWP forecasts.

\section{The continuous equations}
\label{sec:kalnay:continuous_equations}

\section{Numerical discretization of the equations of motion}
\label{sec:kalnay:numerical_discretization}

\section{Introduction to the parameterization of subgrid-scale physical processes}
\label{sec:kalnay:parameterization}

\section{Data assimilation}
\label{sec:kalnay:data_assimilation}

\section{Atmospheric predictability and ensemble forecasting}
\label{sec:kalnay:predictability}
