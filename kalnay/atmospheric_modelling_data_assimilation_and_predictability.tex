\chapter{Atmospheric Modelling, Data Assimilation and Predictability}
\label{ch:kalnay}

The following constitutes the notes that I have made whilst going through Atmospheric Modeling, Data Assimilation and Predictability\citep{kalnay_atmospheric_2003}.

\section{Historical overview of numerical weather prediction}
\label{sec:kalnay:historical_overview}

\subsection{Introduction}
\label{subs:historical_overview:introduction}

The improvement in skill of numerical weather prediction over the last 40 years is due to the following factors:
\begin{itemize}
\item The increased power of supercomputers - this allows for finer numerical resolution, and means that fewer approximations are required in the operational atmospheric models;
\item The improved representation of small-scale physical processes - these include clouds, precipitation, turbulent transfer processes;
\item The use of more accurate methods of data assimilation - these lead to improved initial conditions for the models;
\item The increased availability of data - for example, satellite and aircraft data over the oceans and the Southern Hemisphere. 
\end{itemize}

\subsection{Early developments}
\label{subs:historical_overview:early_developments}

\subsection{Primitive equations, global and regional models, and nonhydrostatic models}
\label{subs:historical_overview:primitive_equations}

\section{The continuous equations}
\label{sec:kalnay:continuous_equations}

\section{Numerical discretization of the equations of motion}
\label{sec:kalnay:numerical_discretization}

\section{Introduction to the parameterization of subgrid-scale physical processes}
\label{sec:kalnay:parameterization}

\section{Data assimilation}
\label{sec:kalnay:data_assimilation}

\section{Atmospheric predictability and ensemble forecasting}
\label{sec:kalnay:predictability}
