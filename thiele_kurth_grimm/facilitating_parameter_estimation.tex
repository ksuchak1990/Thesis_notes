\chapter{Facilitating Parameter Estimation and Sensitivity Analysis of Agent-Based Models: A Cookbook Using NetLogo and R}
\label{ch:thiele_kurther_grimm}

This chapter contains notes that I have taken on Facilitating Parameter Estimation and Sensitivity Analysis of Agent-Based Models: A Cookbook Using NetLogo and R\citep{thiele_facilitating_2014}.

\section*{Abstract}
\label{thiele:absract}

The use of agent-based models (ABMs) to investigate real-world phenomena is becoming more and more frequent.
In these scenarios, it can be of great use to calibrate model parameters using pertinent real-world data sets.
It is also important to undertake sensitivity analysis; this allows us to evaluate the importance of different mechanisms in the model, and its response to uncertainty in parameters.
This paper aims to outline a set of methods for implementing parameter estimation and sensitivity analysis for ABMs.
These methods are outlined with the use of code samples in NetLogo and R.

\section{Introduction}
\label{thiele:introduction}

Agent-based models implement an explicit representation of individual agents and their behaviours; in the context of ABMs, agents can be humans, institutions, organisms, etc.
They are typically used when considering at least one of:
\begin{itemize}
    \item heterogeneity among individuals,
    \item local interactions,
    \item adaptive behaviour based on decision-making.
\end{itemize}
ABMs therefore lend themselves to the investigation of social systems, ecological systems - fields in which they have become established methods.
Becoming established in these fields has occurred in two phases:
\begin{enumerate}
    \item ABMs are designed in research fields with the aim of building representations. 
    Here, the models are developed to demonstrate mechanisms, but not typically with the intention of making predictions.
    The models are most often evaluated qualitative.
    Do these really explain observed phenomena?
    \item 
\end{enumerate}

\section{Parameter estimation and calibration}
\label{thiele:parameter_estimation}

\section{Sensitivity analysis}
\label{thiele:sensitivity_analysis}

