\chapter{Facilitating Parameter Estimation and Sensitivity Analysis of Agent-Based Models: A Cookbook Using NetLogo and R}
\label{ch:thiele_kurther_grimm}

This chapter contains notes that I have taken on Facilitating Parameter Estimation and Sensitivity Analysis of Agent-Based Models: A Cookbook Using NetLogo and R\citep{thiele_facilitating_2014}.

\section*{Abstract}
\label{thiele:absract}

The use of agent-based models (ABMs) to investigate real-world phenomena is becoming more and more frequent.
In these scenarios, it can be of great use to calibrate model parameters using pertinent real-world data sets.
It is also important to undertake sensitivity analysis; this allows us to evaluate the importance of different mechanisms in the model, and its response to uncertainty in parameters.
This paper aims to outline a set of methods for implementing parameter estimation and sensitivity analysis for ABMs.
These methods are outlined with the use of code samples in NetLogo and R.

\section{Introduction}
\label{thiele:introduction}

Agent-based models implement an explicit representation of individual agents and their behaviours; in the context of ABMs, agents can be humans, institutions, organisms, etc.
They are typically used when considering at least one of:
\begin{itemize}
    \item heterogeneity among individuals,
    \item local interactions,
    \item adaptive behaviour based on decision-making.
\end{itemize}
ABMs therefore lend themselves to the investigation of social systems, ecological systems - fields in which they have become established methods.
Becoming established in these fields has occurred in two phases:
\begin{enumerate}
    \item ABMs are designed in research fields with the aim of building representations. 
    Here, the models are developed to demonstrate mechanisms, but not typically with the intention of making predictions.
    The models are most often evaluated qualitative.
    Do these really explain observed phenomena?
    \item Upon reaching a critical mass of models pertaining to a line of enquiry, attention shifts towards using the models to gain insight into how systems actually work.
    With this, we see a tendency towards more quantitative analyses of the models.
    We therefore look to undertake sensitivity analysis and calibration.
\end{enumerate}
The aforementioned quantitative approaches as still not frequently used.
This may be due to a number of reasons:
\begin{itemize}
    \item availability of data,
    \item lack of agreement on the theory of certain processes,
    \item the complexity of agents' decision-making processes.
\end{itemize}
Furthermore, the users of ABMs ecology and social sciences may lack the technical training on the theory and implementation of these sensitivity analysis and calibration methods.
Whilst the methods themselves have been documented, they are typically dense, and consequently not very accessible to practitioners in the fields of ecology and social sciences.

The aim of this article is, therefore, to introduce software and scripts to guide the use of sensitivity analysis, calibration and the design of simulations.
This will be achieved using a combination of R and NetLogo.
A brief overview of these languages, and other pertinent software is provided.

The rigorous backgrounds of the sensitivity analysis and calibration methods is not provided, owing to the already substantial body of literature on the topics.
Furthermore, the provided applications of the methods should be taken as illustrative, and as such have not been fine-tuned or optimised.

\subsection{Software requirements}
\label{thiele:intro:software}

An introduction on how to obtain NetLogo and R is provided, along with a template for running a NetLogo ABM through R (which shall form the basis of subsequent examples).

\subsection{The example model}
\label{thiele:intro:example}

The basis of the model is outlined.
The model implemented is a simplified representation of the group dynamics of a population of territorial bird species with reproductive suppression (that is to say that in each group, the dominant couple suppress the reproduction of subordinate mature members of the population).
The key behaviour of this system that is encapsulated by this model is subordinate birds' decision as to when to leave their territory to undertake scouting forays.
These scouting forays offer opportunities to free alpha positions or breeding positions.
Note that this terminology is not immediately transparent to readers without a ecological simulation background, and could be expanded up further.
Birds that do not successful on their forays return to their home territory.
Birds that are on forays leave themselves open to attack by raptors, and as such experience an increased mortality risk.
The model offers investigators the opportunity to develop new theories regarding the decision to undertake scouting forays.
This is achieved by implementing alternative submodels of the foray decision and comparing the full model output against real-world patterns.
This article implements the simplest submodel, in which the probability of subordinates undertaking scouting forays is constant.

\subsubsection{Purpose}
\label{thiele:intro:example:purpose}

The model is implemented to explore how the decisions of subordinate birds to undertake scouting forays impact the structure and dynamics of the social group and overall population.

\subsubsection{Entities, state variables, and scales}
\label{thiele:intro:example:entities}

The model is made up of the birds and the territories - we consider these to be the entities.
By territory, we mean both the social group of birds, and the space that the group occupies.
The territories that are not occupied by a group are designated as empty.
In this model, our territories are arranged in a one-dimensional row with a periodic boundary, such that exiting one end of the row results in entry at the other end; topologically a ring.
Each of the types of entities consists of a collection of state variables:
\begin{itemize}
    \item Territories:
    \begin{itemize}
        \item coordinates of its location,
        \item list of birds residing at the territory.
    \end{itemize}
    \item Birds:
    \begin{itemize}
        \item sex,
        \item age (in months),
        \item alpha status (i.e. whether or not the bird is alpha).
    \end{itemize}
\end{itemize}
The model is simulated with a time-step of one month, running for 22 years.
The output of the first two years is ignored.

\subsubsection{Process overview and scheduling}
\label{thiele:intro:example:overview}

The following procedure is implemented iteratively, executing once per time-step:
\begin{enumerate}
    \item Update date and ages of birds.
    \item Territories attempt to fill vacant alpha positions. This is achieved by territories that are without an picking the oldest subordinate that fulfils the following criteria as the new alpha:
    \begin{itemize}
        \item subordinate is an adult (age $>$ 12 months),
        \item subordinate is of the correct sex.
    \end{itemize}
    \item Birds undertake scouting forays:
    \begin{itemize}
        \item Adult subordinates decide whether to scout for new territory that has a vacant alpha position.
        \item If no other non-alpha is in the territory then the subordinate adult stays there.
        \item  If there are any older non-alphas present in the current home territory, the subordinate adult undertakes a scouting foray with probability $scout-prob$.
        \item If the bird chooses to scout, then it randomly moves either left or right along the row of territories.
        \item Scouting birds are can only explore up to the $scouting-distance$ in their chosen direction.
        \item Out of those territories, the bird chooses to occupy the the one that  
    \end{itemize}
\end{enumerate}

\subsubsection{Design concepts}
\label{thiele:intro:example:design}

\subsubsection{Initialisation}
\label{thiele:intro:example:intial}

\subsubsection{Input data}
\label{thiele:intro:example:input}

\subsubsection{Submodels}
\label{thiele:intro:example:submodels}

\section{Parameter estimation and calibration}
\label{thiele:parameter_estimation}

\subsection{Preliminaries: Fitting criteria for the example model}
\label{thiele:parameter:prelims}

\subsection{Full factorial design}
\label{thiele:parameter:full}

\subsection{Classical sampling methods}
\label{thiele:parameter:classical}

\subsubsection{Simple random sampling}
\label{thiele:parameter:classical:simple}

\subsection{Latin hypercube sampling}
\label{thiele:parameter:latin}

\subsection{Optimisation methods}
\label{thiele:parameter:optimisation}

\subsubsection{Gradient and quasi-Newton methods}
\label{thiele:parameter:optimisation:gradient}

\subsubsection{Simulated annealing}
\label{thiele:parameter:optimisation:annealing}

\subsubsection{Evolutionary or genetic algorithms}
\label{thiele:parameter:optimisation:genetic}

\subsection{Bayesian methods}
\label{thiele:parameter:bayesian}

\subsubsection{Rejection and regression sampling}
\label{thiele:parameter:bayesian:rejection}

\subsubsection{Markov chain Monte Carlo}
\label{thiele:parameter:bayesian:mcmc}

\subsubsection{Sequential Monte Carlo}
\label{thiele:parameter:bayesian:seqmc}

\subsection{Costs and benefits of approaches to parameter estimation and calibration}
\label{thiele:parameter:cost_and_benefits}

\section{Sensitivity analysis}
\label{thiele:sensitivity_analysis}

\subsection{Preliminaries: Experimental setup for the example model}
\label{thiele:sensitivity:prelims}

\subsection{Local sensitivity analysis}
\label{thiele:sensitivity:local}

\subsection{Screening methods}
\label{thiele:sensitivity:screening}

\subsubsection{Morris's elementary effects screening}
\label{thiele:sensitivity:screening:morris}

\subsection{Global sensitivity analysis}
\label{thiele:sensitivity:global}

\subsubsection{Excursion: Design of Experiment}
\label{thiele:sensitivity:global:excursion}

\subsubsection{Regression-based methods}
\label{thiele:sensitivity:global:regression}

\subsubsection{Partial (rank) correlation coefficient}
\label{thiele:sensitivity:global:partial}

\subsubsection{Standardised (rank) regression coefficient}
\label{thiele:sensitivity:global:standardised}

\subsubsection{Variance decomposition methods}
\label{thiele:sensitivity:global:variance}

\subsubsection{Sobol' Method}
\label{thiele:sensitivity:sobol}

\subsubsection{Extended Fourier amplitude sensitivity test}
\label{thiele:sensitivity:sobol:fourier}

\subsubsection{FANOVA decomposition with Gaussian process}
\label{thiele:sensitivity:sobol:fanova}

\subsection{Costs and benefits of approaches to sensitivity analysis}
\label{thiele:sensitivity:sobol:costs_and_benefits}

\section{Discussion}
\label{thiele:discussion}
