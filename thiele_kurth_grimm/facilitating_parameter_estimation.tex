\chapter{Facilitating Parameter Estimation and Sensitivity Analysis of Agent-Based Models: A Cookbook Using NetLogo and R}
\label{ch:thiele_kurther_grimm}

This chapter contains notes that I have taken on Facilitating Parameter Estimation and Sensitivity Analysis of Agent-Based Models: A Cookbook Using NetLogo and R\citep{thiele_facilitating_2014}.

\section*{Abstract}
\label{thiele:absract}

The use of agent-based models (ABMs) to investigate real-world phenomena is becoming more and more frequent.
In these scenarios, it can be of great use to calibrate model parameters using pertinent real-world data sets.
It is also important to undertake sensitivity analysis; this allows us to evaluate the importance of different mechanisms in the model, and its response to uncertainty in parameters.
This paper aims to outline a set of methods for implementing parameter estimation and sensitivity analysis for ABMs.
These methods are outlined with the use of code samples in NetLogo and R.

\section{Introduction}
\label{thiele:introduction}

Agent-based models implement an explicit representation of individual agents and their behaviours; in the context of ABMs, agents can be humans, institutions, organisms, etc.
They are typically used when considering at least one of:
\begin{itemize}
    \item heterogeneity among individuals,
    \item local interactions,
    \item adaptive behaviour based on decision-making.
\end{itemize}
ABMs therefore lend themselves to the investigation of social systems, ecological systems - fields in which they have become established methods.
Becoming established in these fields has occurred in two phases:
\begin{enumerate}
    \item ABMs are designed in research fields with the aim of building representations. 
    Here, the models are developed to demonstrate mechanisms, but not typically with the intention of making predictions.
    The models are most often evaluated qualitative.
    Do these really explain observed phenomena?
    \item Upon reaching a critical mass of models pertaining to a line of enquiry, attention shifts towards using the models to gain insight into how systems actually work.
    With this, we see a tendency towards more quantitative analyses of the models.
    We therefore look to undertake sensitivity analysis and calibration.
\end{enumerate}
The aforementioned quantitative approaches as still not frequently used.
This may be due to a number of reasons:
\begin{itemize}
    \item availability of data,
    \item lack of agreement on the theory of certain processes,
    \item the complexity of agents' decision-making processes.
\end{itemize}
Furthermore, the users of ABMs ecology and social sciences may lack the technical training on the theory and implementation of these sensitivity analysis and calibration methods.
Whilst the methods themselves have been documented, they are typically dense, and consequently not very accessible to practitioners in the fields of ecology and social sciences.

The aim of this article is, therefore, to introduce software and scripts to guide the use of sensitivity analysis, calibration and the design of simulations.
This will be achieved using a combination of R and NetLogo.
A brief overview of these languages, and other pertinent software is provided.

The rigorous backgrounds of the sensitivity analysis and calibration methods is not provided, owing to the already substantial body of literature on the topics.
Furthermore, the provided applications of the methods should be taken as illustrative, and as such have not been fine-tuned or optimised.

\subsubsection{Software requirements}
\label{thiele:intro:software}

An introduction on how to obtain NetLogo and R is provided, along with a template for running a NetLogo ABM through R (which shall form the basis of subsequent examples).

\subsection{The example model}
\label{thiele:intro:example}

The basis of the model is outlined.
The model implemented is a simplified representation of the group dynamics of a population of territorial bird species with reproductive suppression (that is to say that in each group, the dominant couple suppress the reproduction of subordinate mature members of the population).

\section{Parameter estimation and calibration}
\label{thiele:parameter_estimation}

\section{Sensitivity analysis}
\label{thiele:sensitivity_analysis}

